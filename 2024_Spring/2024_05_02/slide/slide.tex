%%%%%%%%%%%%%%%%%%%%%%%%%%%%%%%%%%%%%%%%%
% Beamer Presentation
% LaTeX Template
% Version 2.0 (March 8, 2022)
%
% This template originates from:
% https://www.LaTeXTemplates.com
%
% Author:
% Vel (vel@latextemplates.com)
%
% License:
% CC BY-NC-SA 4.0 (https://creativecommons.org/licenses/by-nc-sa/4.0/)
%
%%%%%%%%%%%%%%%%%%%%%%%%%%%%%%%%%%%%%%%%%

%----------------------------------------------------------------------------------------
%	PACKAGES AND OTHER DOCUMENT CONFIGURATIONS
%----------------------------------------------------------------------------------------

\documentclass[
	12pt, % Set the default font size, options include: 8pt, 9pt, 10pt, 11pt, 12pt, 14pt, 17pt, 20pt
	%t, % Uncomment to vertically align all slide content to the top of the slide, rather than the default centered
	aspectratio=169, % Uncomment to set the aspect ratio to a 16:9 ratio which matches the aspect ratio of 1080p and 4K screens and projectors
]{beamer}

\graphicspath{{Images/}{./}} % Specifies where to look for included images (trailing slash required)

\usepackage{booktabs} % Allows the use of \toprule, \midrule and \bottomrule for better rules in tables

%----------------------------------------------------------------------------------------
%	SELECT LAYOUT THEME
%----------------------------------------------------------------------------------------

% Beamer comes with a number of default layout themes which change the colors and layouts of slides. Below is a list of all themes available, uncomment each in turn to see what they look like.

%\usetheme{default}
%\usetheme{AnnArbor}
%\usetheme{Antibes}
%\usetheme{Bergen}
%\usetheme{Berkeley}
%\usetheme{Berlin}
%\usetheme{Boadilla}
%\usetheme{CambridgeUS}
%\usetheme{Copenhagen}
%\usetheme{Darmstadt}
%\usetheme{Dresden}
%\usetheme{Frankfurt}
%\usetheme{Goettingen}
%\usetheme{Hannover}
%\usetheme{Ilmenau}
%\usetheme{JuanLesPins}
%\usetheme{Luebeck}
\usetheme{Madrid}
%\usetheme{Malmoe}
%\usetheme{Marburg}
%\usetheme{Montpellier}
%\usetheme{PaloAlto}
%\usetheme{Pittsburgh}
%\usetheme{Rochester}
%\usetheme{Singapore}
%\usetheme{Szeged}
%\usetheme{Warsaw}

%----------------------------------------------------------------------------------------
%	SELECT COLOR THEME
%----------------------------------------------------------------------------------------

% Beamer comes with a number of color themes that can be applied to any layout theme to change its colors. Uncomment each of these in turn to see how they change the colors of your selected layout theme.

%\usecolortheme{albatross}
%\usecolortheme{beaver}
%\usecolortheme{beetle}
%\usecolortheme{crane}
%\usecolortheme{dolphin}
%\usecolortheme{dove}
%\usecolortheme{fly}
%\usecolortheme{lily}
%\usecolortheme{monarca}
%\usecolortheme{seagull}
%\usecolortheme{seahorse}
%\usecolortheme{spruce}
%\usecolortheme{whale}
%\usecolortheme{wolverine}

%----------------------------------------------------------------------------------------
%	SELECT FONT THEME & FONTS
%----------------------------------------------------------------------------------------

% Beamer comes with several font themes to easily change the fonts used in various parts of the presentation. Review the comments beside each one to decide if you would like to use it. Note that additional options can be specified for several of these font themes, consult the beamer documentation for more information.

\usefonttheme{default} % Typeset using the default sans serif font
%\usefonttheme{serif} % Typeset using the default serif font (make sure a sans font isn't being set as the default font if you use this option!)
%\usefonttheme{structurebold} % Typeset important structure text (titles, headlines, footlines, sidebar, etc) in bold
%\usefonttheme{structureitalicserif} % Typeset important structure text (titles, headlines, footlines, sidebar, etc) in italic serif
%\usefonttheme{structuresmallcapsserif} % Typeset important structure text (titles, headlines, footlines, sidebar, etc) in small caps serif

%------------------------------------------------

%\usepackage{mathptmx} % Use the Times font for serif text
\usepackage{palatino} % Use the Palatino font for serif text

%\usepackage{helvet} % Use the Helvetica font for sans serif text
\usepackage[default]{opensans} % Use the Open Sans font for sans serif text
%\usepackage[default]{FiraSans} % Use the Fira Sans font for sans serif text
%\usepackage[default]{lato} % Use the Lato font for sans serif text
\usepackage{filecontents} % used for reference
\usepackage[style=numeric, sorting=none, backend=biber]{biblatex} % used for reference

%----------------------------------------------------------------------------------------
%	REFERENCE
%----------------------------------------------------------------------------------------

\addbibresource{myreference.bib}

%----------------------------------------------------------------------------------------
%	SELECT INNER THEME
%----------------------------------------------------------------------------------------

% Inner themes change the styling of internal slide elements, for example: bullet points, blocks, bibliography entries, title pages, theorems, etc. Uncomment each theme in turn to see what changes it makes to your presentation.

%\useinnertheme{default}
\useinnertheme{circles}
%\useinnertheme{rectangles}
%\useinnertheme{rounded}
%\useinnertheme{inmargin}

%----------------------------------------------------------------------------------------
%	SELECT OUTER THEME
%----------------------------------------------------------------------------------------

% Outer themes change the overall layout of slides, such as: header and footer lines, sidebars and slide titles. Uncomment each theme in turn to see what changes it makes to your presentation.

%\useoutertheme{default}
%\useoutertheme{infolines}
%\useoutertheme{miniframes}
%\useoutertheme{smoothbars}
%\useoutertheme{sidebar}
%\useoutertheme{split}
%\useoutertheme{shadow}
%\useoutertheme{tree}
%\useoutertheme{smoothtree}

%\setbeamertemplate{footline} % Uncomment this line to remove the footer line in all slides
%\setbeamertemplate{footline}[page number] % Uncomment this line to replace the footer line in all slides with a simple slide count

\setbeamertemplate{navigation symbols}{} % Uncomment this line to remove the navigation symbols from the bottom of all slides
\setbeamertemplate{caption}[numbered] % numbering the figure in preentation

%----------------------------------------------------------------------------------------
%	MACRO
%----------------------------------------------------------------------------------------

\newcommand\meetingdatecompact{Paper Review 05/02}
\newcommand\meetingdate{May 2, 2024}

%----------------------------------------------------------------------------------------
%	PRESENTATION INFORMATION
%----------------------------------------------------------------------------------------

\title[\meetingdatecompact]{GPGait: Generalized Pose-based Gait Recognition \\ \meetingdatecompact} % The short title in the optional parameter appears at the bottom of every slide, the full title in the main parameter is only on the title page

% \subtitle{Optional Subtitle} % Presentation subtitle, remove this command if a subtitle isn't required

\author[Bo Han, Chen]{Bo Han, Chen} % Presenter name(s), the optional parameter can contain a shortened version to appear on the bottom of every slide, while the main parameter will appear on the title slide

\institute[NYCU]{National Yang Ming Chiao Tung University, Taiwan \\ \smallskip \textit{bhchen312551074.cs12@nycu.edu.tw}} % Your institution, the optional parameter can be used for the institution shorthand and will appear on the bottom of every slide after author names, while the required parameter is used on the title slide and can include your email address or additional information on separate lines

\date[\meetingdate]{\meetingdate} % Presentation date or conference/meeting name, the optional parameter can contain a shortened version to appear on the bottom of every slide, while the required parameter value is output to the title slide

%----------------------------------------------------------------------------------------

\begin{document}

%----------------------------------------------------------------------------------------
%	TITLE SLIDE
%----------------------------------------------------------------------------------------

\begin{frame}
	\titlepage % Output the title slide, automatically created using the text entered in the PRESENTATION INFORMATION block above
\end{frame}

%----------------------------------------------------------------------------------------
%	TABLE OF CONTENTS SLIDE
%----------------------------------------------------------------------------------------

% The table of contents outputs the sections and subsections that appear in your presentation, specified with the standard \section and \subsection commands. You may either display all sections and subsections on one slide with \tableofcontents, or display each section at a time on subsequent slides with \tableofcontents[pausesections]. The latter is useful if you want to step through each section and mention what you will discuss.

% \begin{frame}
% 	\frametitle{Presentation Overview} % Slide title, remove this command for no title
	
% 	\tableofcontents % Output the table of contents (all sections on one slide)
% 	%\tableofcontents[pausesections] % Output the table of contents (break sections up across separate slides)
% \end{frame}

%----------------------------------------------------------------------------------------
%	PRESENTATION BODY SLIDES
%----------------------------------------------------------------------------------------

\section{Info}

\begin{frame}
	\frametitle{Info}

	\begin{itemize}
		\item Title: GPGait: Generalized Pose-based Gait Recognition \cite{fu2023gpgait}
		\item Authors: Yang Fu, Shibei Meng, Saihui Hou, Xuecai Hu, Yongzhen Huang
		\item Conference: IEEE/CVF International Conference on Computer Vision
		\item Year: 2023
	\end{itemize}
\end{frame}

\section{Motivation}

\begin{frame}
	\frametitle{Motivation}

	\begin{itemize}
		\item Poor generalization ability of pose-based methods
		\item Performance degradation from unseen environments
		\item Cross-domain factors
		\begin{itemize}
			\item Scale variation (distance from camera)
			\item Tilt view (deployment of camera)
		\end{itemize}
	\end{itemize}

\end{frame}

%------------------------------------------------

\section{Proposed Method}

\begin{frame}
	\frametitle{Proposed Method}

	\begin{itemize}
		\item Human-Oriented Transformation (HOT)
		\begin{itemize}
			\item Obtain unified representation
		\end{itemize}
		\item Human-Oriented Descriptors (HOD)
		\begin{itemize}
			\item Generate invariant features of bone and angle
			\item Capture body structure and gait motion
		\end{itemize}
		\item Part-Aware Graph Convolutional Network (PAGCN)
		\begin{itemize}
			\item Graph partitioning based on body structure
			\item Capture local and global information with different masks
		\end{itemize}
	\end{itemize}

\end{frame}

\begin{frame}
	\frametitle{Proposed Method}
	\framesubtitle{Human-Oriented Transformation (HOT)}

	\begin{figure}
		\centering
		\includegraphics[width=0.8\linewidth]{"./Images/hot.png"}
		\caption{HOT Overview}
	\end{figure}
\end{frame}

\begin{frame}
	\frametitle{Proposed Method}
	\framesubtitle{Human-Oriented Transformation (HOT)}

	\begin{itemize}
		\item Affine transformation
		\begin{itemize}
			\item Take spine angle as reference
			\item neck position \& hip joint
			\item Applied if angle $\theta$ is larger than threshold
		\end{itemize}
		\item Rescale
		\begin{itemize}
			\item Height normalization
		\end{itemize}
		\item Alignment
		\begin{itemize}
			\item Subtract neck joint coordinate
			\item Remove offsets from camera positioning
		\end{itemize}
	\end{itemize}
\end{frame}

\begin{frame}
	\frametitle{Proposed Method}
	\framesubtitle{Human-Oriented Descriptors (HOD)}
	
	% place text and image side by side
	\begin{columns}
		\column{0.55\textwidth}
		\begin{itemize}
			\item Bone features
			\begin{itemize}
				\item Calculated by adjacent joints
			\end{itemize}
			\item Angle features
			\begin{itemize}
				\item Inner angle
				\begin{itemize}
					\item Angular changes inside body
				\end{itemize}
				\item Peripheral angle
				\begin{itemize}
					\item Movement of skeleton edge
				\end{itemize}
			\end{itemize}
		\end{itemize}
		\column{0.45\textwidth}
		\begin{figure}
			\centering
			\includegraphics[width=0.65\linewidth]{"./Images/hod.png"}
			\caption{HOD Overview}
		\end{figure}
	\end{columns}

\end{frame}

\begin{frame}
	\frametitle{Proposed Method}
	\framesubtitle{Part-Aware Graph Convolutional Network (PAGCN)}
	
	\begin{itemize}
		\item Three parameter branches
		\begin{itemize}
			\item joint
			\item Bone
			\item Angle
		\end{itemize}
		\item Local PAGCN
		\begin{itemize}
			\item Partition mask
			\item Capture fine-grained body information
		\end{itemize}
		\item Global PAGCN
		\begin{itemize}
			\item Capture global spatial keypoint relationship
		\end{itemize}
	\end{itemize}

\end{frame}

\begin{frame}
	\frametitle{Proposed Method}
	\framesubtitle{Part-Aware Graph Convolutional Network (PAGCN)}
	
	\begin{figure}
		\centering
		\includegraphics[width=0.55\linewidth]{"./Images/mask_partition.png"}
		\caption{Mask Partition}
	\end{figure}
\end{frame}

%------------------------------------------------

\section{Experiments}

\begin{frame}
	\frametitle{Experiments}
	\framesubtitle{Result}

	\begin{figure}
		\centering
		\includegraphics[width=0.8\linewidth]{"./Images/cross_domain_result.png"}
		% \caption{Cross-domain Results}
	\end{figure}
\end{frame}

\begin{frame}
	\frametitle{Experiments}
	\framesubtitle{Ablation Study}

	\begin{figure}
		\centering
		\includegraphics[width=0.8\linewidth]{"./Images/hot_ablation.png"}
		% \caption{CASIA-B Results}
	\end{figure}
\end{frame}

\begin{frame}
	\frametitle{Experiments}
	\framesubtitle{Ablation Study}

	\begin{figure}
		\centering
		\includegraphics[width=0.8\linewidth]{"./Images/hot_ablation_2.png"}
		% \caption{OUMVLP Results}
	\end{figure}
\end{frame}

\begin{frame}
	\frametitle{Experiments}
	\framesubtitle{Ablation Study}

	\begin{itemize}
		\item Multi-Branch vs. Single-Branch
		\begin{itemize}
			\item Focus on different type of features
		\end{itemize}
		\item With mask partition vs. Without mask partition
		\begin{itemize}
			\item Able to capture fine-grained features
		\end{itemize}
	\end{itemize}
\end{frame}

%------------------------------------------------

\section{Conclusion}

\begin{frame}
	\frametitle{Conclusion}

	\begin{itemize}
		\item Performance on source-domain
		\item vs. silhouette-based methods
		\begin{itemize}
			\item Keypoint vs. Body shape
			\item Robustness to wearing and carrying, etc
		\end{itemize}
		\item Compared to last reviewed paper
		\begin{itemize}
			\item temporal information
			\item computational cost
		\end{itemize}
	\end{itemize}
\end{frame}

%------------------------------------------------

\section{References}

\begin{frame}
	\frametitle{References}

	\printbibliography
\end{frame}

%----------------------------------------------------------------------------------------
%	CLOSING SLIDE
%----------------------------------------------------------------------------------------

\section{Q \& A}

\begin{frame}
    \frametitle{Q \& A}
	\begin{center}
		{\Huge Thanks for Listening}
		
		\bigskip\bigskip % Vertical whitespace
		
		{\LARGE Q \& A}

		% \bigskip\bigskip
		% Contact Me: b083040012@g-mail.nsysu.edu.tw
	\end{center}
\end{frame}

%----------------------------------------------------------------------------------------

\end{document} 